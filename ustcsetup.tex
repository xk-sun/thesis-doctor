\ustcsetup{
  title              = {实时以太网POWERLINK在加速器控制系统中的应用研究},
  title*             = {Application and Research of Ethernet POWERLINK in Accelerator Control System},
  author             = {孙晓康},
  author*            = {Xiaokang Sun},
  speciality         = {核科学与技术},
  speciality*        = {Nuclear Science and Technology},
  supervisor         = {刘功发 教授级高工},
  supervisor*        = {Prof. Gongfa Liu},
  % date               = {2017-05-01},  % 默认为今日
  % professional-type  = {专业学位类型},
  % professional-type* = {Professional degree type},
  % secret-level       = {秘密},     % 绝密|机密|秘密,注释本行则不保密
  % secret-level*      = {Secret},  % Top secret|Highly secret|Secret
  % secret-year        = {10},      % 保密年限
}


% 加载宏包
\usepackage{graphicx}
\graphicspath{{figures/}}
\usepackage{booktabs}
\usepackage{longtable}
\usepackage[ruled,linesnumbered]{algorithm2e}
\usepackage{siunitx}
\usepackage{amsthm}
\usepackage{listings}
\usepackage{amsmath}
\usepackage{multirow}
\usepackage{diagbox}

% 数学命令
\input{math-commands.tex}

% 配置图片的默认目录
\graphicspath{{figures/}}

% 用于写文档的命令
\DeclareRobustCommand\cs[1]{\texttt{\char`\\#1}}
\DeclareRobustCommand\pkg{\textsf}
\DeclareRobustCommand\file{\nolinkurl}


% hyperref 宏包在最后调用
\usepackage[colorlinks,linkcolor=blue]{hyperref}
\usepackage{listings}
\lstset{
  columns=fixed,       
  numbers=left,                                        % 在左侧显示行号
  numberstyle=\tiny\color{gray},                       % 设定行号格式
  frame=none,                                          % 不显示背景边框
  backgroundcolor=\color[RGB]{245,245,244},            % 设定背景颜色
  keywordstyle=\color[RGB]{40,40,255},                 % 设定关键字颜色
  numberstyle=\footnotesize\color{darkgray},           
  commentstyle=\it\color[RGB]{0,96,96},                % 设置代码注释的格式
  %  stringstyle=\rmfamily\slshape\color[RGB]{128,0,0},   % 设置字符串格式
  stringstyle=\ttfamily\slshape\color[RGB]{128,0,0},   % 设置字符串格式
  showstringspaces=false,                              % 不显示字符串中的空格
  language=python,                                       % 设置语言
  basicstyle=\footnotesize\ttfamily,
}
\usepackage{xcolor}

\usepackage{verbatim}
\usepackage{subfigure} 
\usepackage{multirow}
\usepackage{emptypage}
%\usepackage{enumerate}
\usepackage{enumitem}
\usepackage{float}
