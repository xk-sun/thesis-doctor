% !TeX root = ../main.tex

\begin{acknowledgements}

本论文从动笔到完成历时约半年的时间,期间因为新冠肺炎疫情,我无法返校,这篇论文的五分之四部分是在家里写就的,遭遇很多困难,诸如疫情期间居家隔离造成的焦虑、查询数据资料的不便、论文写作方向上的困惑等等。在此期间很多人给我提供了帮助,借此机会,我想感谢他们。

首先,我要感谢我的导师刘功发教授。回顾五年来的研究生生涯,刘老师一直对我严格要求和耐心指导,刘老师渊博的专业知识和严谨的治学态度使我受益匪浅。特别是在撰写论文的过程中,不管刘老师多忙,总会抽出时间来帮我修改论文,经常和我讨论论文到深夜,非常辛苦,谢谢刘老师。师者,传道授业解惑也。除了在学习科研上对我提供的帮助之外,刘老师待人温和谦虚,对事认真负责,这都令我我受益终身。能够刘老师的指导下完成研究生阶段的学习是我的运气,这五年来谢谢您的指导与关怀。

感谢李京\hbox{\scalebox{0.7}[1]{礻}\kern-.3em\scalebox{0.6}[1]{韦}}老师在本课题研究过程中对我的指导,特别是在HALF设备保护系统的设计部分和英文写作部分都帮助了我很多。在您一次次的指导中,我不仅学到了专业知识,更提高了科研素养。本课题的顺利完成离不开您的无私帮助。

感谢李川、宣科和王季刚等各位老师对我的研究生学习阶段的教导和鼓励,老师们平易近人,有忙必帮,谢谢你们的热心帮助。

感谢宋一凡、黄子滪、康豪三位同门师兄对我的帮助,特别是宋一凡、黄子滪两位师兄,可以说是尽师兄所能的在学习和生活中给予我帮助,有你们这种师兄是我的福气。感谢控制组各位同学:万昆、谢正源、徐双、陈鑫、辛尚聪、王子建、张震、汪冠良、林广、秦天、于永波、翟港佳对我的帮助,有你们的控制组很团结,气氛活跃,在控制组的每一天心情都很愉悦。感谢邵琢瑕、杜百廷两位同学的陪伴,和你们一起经历了太多,每件事都很开心,都是人生中最珍贵的回忆。特别感谢王思伟师兄对我英文写作方面的帮助。感谢张通、刘刚文、唐运盖、孔二帅等多位师兄和同学,感谢你们对我的照顾和帮助。

感谢我的爸妈对无限的支持,特别是疫情期间居家隔离,一切以我写论文为中心,给我创造了良好的学习环境。感谢远在北京的杜安妮同学对我的鼓励和支持,谢谢你这五年的陪伴。

博士学习生涯即将结束,这五年来我成长太多,再次感谢关心我和帮助我的朋友们。


\end{acknowledgements}
