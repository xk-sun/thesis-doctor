% !TeX root = ../main.tex

\chapter{总结与展望}

\section{总结}
Ethernet POWERLINK作为一种开源实时以太网技术广泛应用于工业控制领域,特别是有高实时性需求的场合,例如高性能的同步运动控制应用。在粒子加速器控制领域,EPICS是目前国际上应用最广泛的控制系统。论文将POWERLINK以太网技术和EPICS环境结合起来,开展了一系列的应用研究工作,现将研究工作总结如下:

(1)对POWERLINK通信协议进行了分析和性能测试。openPOWERLINK是POWERLINK协议栈的开源实现版本,我们基于openPOWERLINK搭建了2个由不同硬件组成的测试系统,测试系统的硬件组成分别是2台RT-Linux PC、1台RT-Linux PC和6块DE2-115 FPGA开发板。在这2个测试平台上分别进行了POWERLINK通信实验,采用网络分析仪netANALYZER抓取了POWERLINK数据帧,并利用Wireshark软件对通信过程进行诊断分析,掌握了POWERLINK协议的数据帧结构和通信机制。同时测试了不同硬件平台下POWERLINK协议的实时性能,2台RT-Linux PC的千兆POWERLINK通信周期最快为1ms,由RT-Linux PC和DE2-115 FPGA开发板组成的百兆POWERLINK系统通信周期最快为700$\mu$s,POWERLINK协议在FPGA硬件平台下的实时性能明显优于其在RT-Linux PC平台下的性能。采用理论分析的方法,推导出了线型拓扑结构下POWERLINK多节点系统的通信周期计算公式。网络模拟基于OMNet++仿真器展开,建立了支持POWERLINK通信的主从节点模型。

(3)设计了EPICS环境下基于POWERLINK的分布式IO系统。我们提出了两套系统方案,分别是主站PC方案和全站FPGA方案。主站PC方案的系统通信网络是千兆POWERLINK,各节点依次相连成菊花链拓扑结构,从站采用基于Zynq的控制器,负责信号采集和输出,主站负责收集处理各从站IO数据并传输至EPICS IOC。我们开发了基于Zynq芯片的从站控制器,Zynq的FPGA部分实现了POWERLINK协议的物理层和数据链路层,并且实现了HUB的功能,ARM部分实现了POWERLINK协议的应用层。主站PC方案采用一台RT-Linux PC作为系统主站,PC上运行了IOC应用程序和内核空间下的openPOWERLINK程序,我们基于进程间Socket开发了EPICS设备驱动程序。为测试系统性能,我们搭建了10个节点组成的测试系统。测试系统的通信周期最快可到275$\mu$s,控制器本地响应时间约为400$\mu$s,系统全局响应时间为870$\mu$s。经研究发现,本地响应时间主要是由从站输入/输出接口电路中的光耦合器延时;同时作为主站的RT-Linux PC的响应时间较慢,约为16.296$\mu$s。

基于主站PC方案系统的测试结果,我们提出了全站FPGA方案。在全站FPGA方案中,从站控制器的输入/输出接口电路采用高速磁隔离器,主站采用Zynq控制器来实现,主站与运行在PC上的IOC应用程序通过UDP Socket的方式通信。为测试改进方案的系统性能,我们搭建了5个节点组成的系统,通信周期最快可到50$\mu$s,控制器本地响应时间为5$\mu$s,系统全局响应时间为160$\mu$s,全站FPGA方案系统的实时性能明显优于主站PC方案系统。我们基于全站FPGA方案系统的测试结果,进一步完善了理论计算方法,然后通过对全站FPGA方案系统进行仿真模拟,模拟得到的系统通信周期51.1$\mu$s,接近实际测试系统的50$\mu$s通信周期,验证了仿真建模分析方法的可行性,


(4)目前国家同步辐射实验室正在进行HALF预研工程,我们基于POWERLINK设计了HALF设备保护系统(EPS)。HALF EPS由注入器分总体EPS和储存环分总体EPS组成,各分总体EPS基于独立的千兆POWERLINK网络设计,主站与从站均采用Zynq控制器,从站将设备联锁信号通过POWERLINK上传至主站,经过主站的联锁逻辑处理之后,主站通过POWERLINK网络将保护命令传输至相应的从站实施保护。注入器分总体EPS可由1台联锁主站和13台联锁从站组成,我们详细分析注入器EPS的联锁保护逻辑,并统计了注入器EPS的联锁信号数量。根据注入器分总体EPS的设计规模,采用理论分析和网络仿真两种方法对其实时性能进行了评估,得到最大响应时间的的估算结果分别为829.996$\mu$s和819.0188$\mu$s,均完全满足设备保护系统响应时间10ms响应时间的需求。储存环分总体EPS可由1台联锁主站和20台联锁从站组成,我们详细分析储存环EPS各系统的联锁保护逻辑,并统计了储存环分总体EPS的联锁信号数量。根据储存环分总体EPS的设计规模,采用理论分析和网络仿真两种方法对其实时性能进行了评估,得到最大响应时间的的估算结果分别为1.7ms和1.671ms,也完全满足10ms响应时间的需求。最后基于Archive Appliance设计了HALF EPS的历史数据存档与查询系统,基于Phoebus/Alarms设计了HALF EPS报警系统。



\section{展望}

本论文的研究内容还有一些有待完善和扩展的地方,具体如下:

(1)除了实时性,还需要开展POWERLINK冗余技术的研究,增加HALF设备保护系统的可靠性。POWERLINK技术提供了三种冗余实现方式:双网冗余、环型冗余和多主冗余。双网冗余是在系统中采用两个独立的网络,当一个网络出现故障,另一个网络依然可以工作,不会丢失数据桢。环型冗余是一种常用的冗余,将菊花链拓扑结构的最后一个节点再与主站相连接以构成了一个环。当其中一根线缆出现问题,这个系统依然可以继续工作。多主冗余就是在一个系统中,存在多个主站,仅有主站一个处于活动状态,其他的主站处于备用状态,当活动主站出现故障时,备用主站接替其工作,继续维持网络的稳定运行。

(2)在加速器控制领域,除了设备联锁保护系统,POWERLINK技术还可应用于电子储存环轨道快反馈(Fast Orbit Feedback,FOFB)系统。FOFB系统负责对储存环的束流闭轨畸变进行校正,快反馈校正频率要求达到KHz量级。在一个闭轨校正周期中,系统
需要完成束流位置数据的传输、轨道校正的计算和校正铁电源电流值的设定。FOFB系统由若干个遍布储存环的校正单元组成,我们可以采用Zynq控制器作为FOFB系统的校正单元控制器,控制器负责接收来自全环的束流位置数据,按照闭轨校正算法计算校正铁电流值,电流值通过POWERLINK传输至各校正铁电源控制器\cite{Tian-2015}。目前正在与电源组合作共同开展此项工作\cite{Shao2019}。

