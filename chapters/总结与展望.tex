% !TeX root = ../main.tex

\chapter{总结与展望}

\section{总结}
Ethernet POWERLINK作为一种开源实时以太网技术已广泛应用于工业控制领域,特别是有高实时性需求的场合,例如高性能的同步运动控制应用,但是在加速器控制领域,与POWERLINK相关的研究和应用还很少。EPICS作为加速器控制领域中应用最广泛的控制系统开发工具,目前还未见与POWERLINK相关的应用与研究。本论文将POWERLINK以太网技术和EPICS环境结合起来,开展了一系列的应用研究工作,总结如下:

(1)对POWERLINK通信协议进行了分析和性能测试。openPOWERLINK是POWERLINK协议栈的开源实现版本,可在x86、FPGA、PLC等硬件平台上实现。我们基于openPOWERLINK搭建了2种测试系统,第1种测试系统由2台RT-Linux PC组成,PC之间基于千兆POWERLINK通信,第2种测试系统由1台RT-Linux PC和6块DE2-115 FPGA开发板组成,节点之间通过百兆POWERLINK通信。基于这2种测试系统,我们分别进行了POWERLINK通信实验,采用网络分析仪netANALYZER抓取了POWERLINK数据帧,并利用Wireshark软件对通信过程进行诊断分析,掌握了POWERLINK协议的数据帧结构和通信机制。我们还测试了两套系统的POWERLINK通信周期,第一种系统的通信周期最快为1ms,第二种系统的通信周期最快为700$\mu$s,POWERLINK协议在FPGA硬件平台上的实时性能明显优于其在RT-Linux PC平台上的性能。我们还根据测试系统的实测通信参数,发展了理论计算和仿真建模两种方法来估算POWERLINK系统的通信周期。在实际应用中,节点数会达到几十个,甚至更多。通信周期是系统设计中最重要的参数,通过理论计算和仿真建模方法对系统的通信周期进行估算,可为系统的设计提供依据。

(2)设计了EPICS环境下基于POWERLINK的分布式IO系统。工业控制领域中大都是采用百兆POWERLINK作为通信网络,为了进一步提高实时性能,我们设计了一套基于千兆POWERLINK的分布式IO系统方案。方案的主站是一台RT-Linux PC,PC上运行了IOC应用程序和内核空间下的openPOWERLINK主站程序,我们基于进程间Socket通信开发了相应的EPICS设备驱动程序。从站采用基于Zynq的控制器,Zynq芯片的FPGA实现了POWERLINK协议的物理层和数据链路层,并且实现了HUB的功能,ARM实现了POWERLINK协议的应用层。为测试系统性能,我们搭建了由1个主站和10个从站组成的测试系统。测试系统的通信周期最快可到275$\mu$s,控制器本地响应时间约为400$\mu$s,系统全局响应时间为870$\mu$s。经研究发现,本地响应时间主要是由从站输入/输出接口电路中的光耦合器延时造成的;作为主站的RT-Linux PC的响应时间约为16.168$\mu$s。

通过对系统测试结果的分析,我们针对光耦延时和主站响应延时设计了相应的改进方案。改进方案的从站输入/输出接口电路采用ADuM1400数字隔离器来缩短信号处理的延时,主站采用Zynq控制器来实现,与运行在PC上的IOC应用程序通过UDP Socket的方式通信。我们搭建了由1个主站和5个从站组成的测试系统来测试改进方案的性能,系统的通信周期最快可到50$\mu$s,从站的本地响应时间为5$\mu$s,系统全局响应时间为160$\mu$s,作为主站的zynq控制器响应时间约为1.717$\mu$s,改进方案的实时性能明显得到提升。我们根据改进方案系统的实测结果,进一步完善了理论计算和仿真建模方法,从而为POWERLINK系统应用设计提供理论依据。


(3)目前国家同步辐射实验室正在进行HALF预研工程,我们基于全站FPGA方案设计了HALF设备保护系统(EPS)。HALF EPS由注入器分总体EPS和储存环分总体EPS组成,各分总体EPS基于独立的千兆POWERLINK网络设计,从站将设备联锁信号通过POWERLINK上传至主站,经过主站的联锁逻辑处理之后,主站通过POWERLINK网络将保护命令传输至相应的从站实施保护。注入器分总体EPS可由1台联锁主站和13台联锁从站组成,我们详细分析注入器EPS的联锁保护逻辑,并统计了注入器EPS的联锁信号数量。根据注入器分总体EPS的设计规模,采用理论计算和仿真模拟两种方法对其实时性能进行了预估,得到最大响应时间的的估算结果分别为802.100$\mu$s和798.184$\mu$s,均完全满足设备保护系统10ms响应时间的设计指标。储存环分总体EPS可由1台联锁主站和20台联锁从站组成,我们详细分析储存环EPS各系统的联锁保护逻辑,并统计了储存环分总体EPS的联锁信号数量。根据储存环分总体EPS的设计规模,采用理论计算和仿真模拟两种方法对其实时性能进行了预估,得到最大响应时间的的估算结果分别为1.643ms和1.634ms,也完全满足10ms响应时间的设计指标。最后基于Archive Appliance设计了HALF EPS的历史数据存档与查询系统,基于Phoebus/Alarms设计了HALF EPS报警系统。



\section{展望}

本论文的研究内容还有一些有待完善和扩展的地方,具体如下:

(1)本论文研究工作中基于Zynq的千兆POWERLINK控制器由贝加莱工业自动化中国有限公司开发,其中基于FPGA的POWERLIN协议实现部分不是开源的,这限制了我们进行接口扩展等二次开发工作。在后续的工作中,我们可以深入研究基于FPGA实现POWERLINK协议部分的内容,从而可以根据实际需求进行自主开发。

(2)本文对百兆和千兆POWERLINK的实时性能进行了研究,目前POWERLINK也可以支持万兆以太网,但是需要专用的硬件设备和硬件驱动程序。未来我们可以进行万兆POWERLINK的相关研究,从而进一步提高POWERLINK协议的实时性能。当万兆POWERLINK系统的响应时间可以达到百微秒量级时,可以尝试将万兆POWERLINK技术应用于电子储存环轨道快反馈系统等更多加速器控制场合\cite{Shao2019,Tian-2015}。

(3)本文的研究工作主要集中在POWERLINK的实时性方面,后续还可以开展POWERLINK可靠性方面的研究,例如POWERLINK冗余技术的研究。POWERLINK提供了三种冗余实现方式:双网冗余、环型冗余和多主冗余。双网冗余是在系统中采用两个独立的网络,当一个网络出现故障,另一个网络依然可以工作,不会丢失数据帧。环型冗余是一种常用的冗余,将菊花链拓扑结构的最后一个节点再与主站相连接以构成了一个环。当其中一根线缆出现问题,这个系统依然可以继续工作。多主冗余就是在一个系统中存在多个主站,仅有主站一个处于活动状态,其他的主站处于备用状态,当活动主站出现故障时,备用主站接替其工作,继续维持网络的稳定运行。


